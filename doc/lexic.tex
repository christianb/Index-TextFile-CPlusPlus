\section{Lexic}
Die Klasse Lexic soll für eine lexikografische Sortierung in der Map der Klasse Index sorgen. Ein einfacher vergleich zweier Zeichen reicht nicht aus, da hier nur der ASCII Wert verglichen wird. Die Funktion lexic_compare() vergleicht zwar auch zwischen Groß und Klein SChreibung allerdings werden z.B. Wörter wie "hallo" und "Hallo" als Identisch betrachtet und tauchen dann in der Map nur als eine der beiden Varianten auf. Je nachdem welche zuerst in der Map war. Beide Varianten liefern also keine zufriedenstellende Lösung an die Anforderungen der Aufgabe. Daher wurde eine eigene Implementierung einer Lexikografischen Sortierung erstellt. Diese funktioniert wie folgt.
\subsection{Funktion}
Die Klasse Lexic stellt lediglich einen Funktionsoperator operator()(string s1, string s2) zur Verfügung. Diesem werden zwei Strings zum Vergleich übergeben. Nun werden die Strings Zeichenweise verglichen, beginnend bei dem ersten Zeichen. Zuerst findet ein case insensitive Vergleich aller Zeichen statt. 
\subsubsection{Case Insensitive}
Dazu werden die Zeichen in Kleinbuchstaben mit Hilfe der Methode tolower(char c) verwandelt. Ist ein Zeichen kleiner oder größer ist der Vergleich vorbei. Denn nun kann eindeutig gesagt werden ob der Buchstabe vor oder nach dem anderen Buchstaben im Alphabet kommt. sind beide Zeichen jedoch gleich wird zum nächsten Zeichen der beiden strings gesprungen. Beide Strings werden so lange untersucht bis ein Buchstabe kleiner oder Größer ist. Ist einer der beiden Strings zuende kann nicht mehr verglichen werden. Bis hier hin kann also gesagt werden das entweder beide Strings identisch sind, oder einer von beiden länger ist aber den anderen bis hier hin als Teilstring enthält. z.B. "Welt" und "Weltkarte". Nun wird also untersucht ob beide Strings gleich lang sind. Wenn nicht so ist der kürzere String kleiner als der längere. Sind beide Strings jedoch gleich lang muss eine case sensitive Untersuchung stattfinden. 
\subsubsection{Case Sensitive}
Beide Strings werden also wieder von Anfang bis Ende Zeichenweise untersucht. Für eine Case Sensitive Untersuchung wird einfach der ASCII Wert der beiden Zeichen verglichen. Ist der Buchstabe ein Großbuchstabe, so kommt er vor dem Kleinbuchstaben. Dies kann aber auch umgetauscht werden. Wir haben uns dafür entschieden das ein Großbuchstabe vor einem gleichen Kleinbuchstaben kommt.
Dieser Vergleich findet so lange statt bis ein Buchstabe größer oder kleiner ist. Sind beide Strings zuende (denn beide sind hier gleich lang) sind alle Buchstaben bezüglich der Groß- Kleinschreibung gleich. In diesem Fall liegen also zwei absolut identische Wörter vor. 